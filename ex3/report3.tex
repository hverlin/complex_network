%% LyX 2.2.2 created this file.  For more info, see http://www.lyx.org/.
%% Do not edit unless you really know what you are doing.
\documentclass[english]{article}
\usepackage[T1]{fontenc}
\usepackage[latin9]{inputenc}
\usepackage{geometry}
\geometry{verbose,tmargin=2cm,bmargin=2cm,lmargin=2cm,rmargin=2cm}
\usepackage{float}
\usepackage{amsmath}
\usepackage{amsthm}
\usepackage{graphicx}

\makeatletter
%%%%%%%%%%%%%%%%%%%%%%%%%%%%%% Textclass specific LaTeX commands.
\numberwithin{equation}{section}
\numberwithin{figure}{section}

%%%%%%%%%%%%%%%%%%%%%%%%%%%%%% User specified LaTeX commands.
\renewcommand{\thesubsection}{\thesection. \alph{subsection})}
\renewcommand{\thesubsubsection}{-}

\makeatother

\usepackage{babel}
\begin{document}

\title{CS-E5740 - Complex Networks\\
Exercise set 3}

\author{Hugues Verlin (584788)\\
\texttt{hugues.verlin@aalto.fi}}
\maketitle

\section{Degree correlations and assortativity}

\subsection{Create a scatter plot of the degrees of nodes incident to each edge}

\begin{figure}[H]
\begin{centering}
\includegraphics[width=0.6\paperwidth]{ex3/graphs/scatter_fb}
\par\end{centering}
\caption{Scatter plot - facebook}
\end{figure}
\begin{figure}[H]
\begin{centering}
\includegraphics[width=0.6\paperwidth]{ex3/graphs/scatterkarate_club_network}
\par\end{centering}
\caption{Scatter plot - karate club}
\end{figure}


\subsection{Heat Map}

\begin{figure}[H]
\begin{centering}
\includegraphics[width=0.6\paperwidth]{\string"ex3/graphs/heatmapfacebook wosn\string".pdf}
\par\end{centering}
\caption{Scatter plot - facebook}
\end{figure}
\begin{figure}[H]
\begin{centering}
\includegraphics[width=0.6\paperwidth]{ex3/graphs/heatmapkarate_club_network}
\par\end{centering}
\caption{Scatter plot - karate club}
\end{figure}


\subsection{Assortativity coefficient}

\subsubsection{Results:}

\paragraph{Facebook}
\begin{verbatim}
Own assortativity for facebook wosn: 0.0559847847659
NetworkX assortativity for facebook wosn: 0.0559847847659
\end{verbatim}

\paragraph{Karate Club}
\begin{verbatim}
Own assortativity for karate club network: -0.475613097685
NetworkX assortativity for karate club network: -0.475613097685
\end{verbatim}
\selectlanguage{english}%

\subsubsection{Comments:}

\inputencoding{latin1}In general, the assortitive value is higher
for graph that are social network. The karate club network is not
a social friendship, thus it has a low assortative value. 

The Facebook network has still a not high value, but it could be explain
by the fact that Facebook is not a real social graph, as you can have
way more friends on Facebook than in the real life.
\selectlanguage{english}%

\subsection{scatter plot of $k_{nn}$ as a function of $k$}

\inputencoding{latin9}\begin{figure}[H]
\begin{centering}
\includegraphics[width=0.6\paperwidth]{\string"ex3/graphs/nearest_neighborsfacebook wosn\string".pdf}
\par\end{centering}
\caption{Scatter plot - facebook}
\end{figure}
\begin{figure}[H]
\begin{centering}
\includegraphics[width=0.6\paperwidth]{ex3/graphs/nearest_neighborskarate_club_network}
\par\end{centering}
\caption{Scatter plot - karate club}
\end{figure}


\section{Centrality measures for undirected networks}

\subsection{Centrality measures}

\subsubsection{Degree}
\begin{itemize}
\item $k(A)=2$
\item $k\left(B\right)=3$
\item $k\left(C\right)=4$
\item $k\left(D\right)=1$
\item $k\left(E\right)=2$
\end{itemize}

\subsubsection{Betweenness centrality}

\begin{align*}
\text{bc}(a) & =\frac{1}{\left(N-1\right)\left(N-2\right)}\sum_{s\neq a}\sum_{t\neq a}\frac{\sigma_{sat}}{\sigma_{st}}\\
 & =\frac{1}{\left(5-1\right)\left(5-2\right)}\times0=0
\end{align*}

\begin{align*}
\text{bc}(b) & =\frac{1}{12}\left(\frac{1}{2}\right)=\frac{1}{24}
\end{align*}

\begin{align*}
\text{bc}(c) & =\frac{1}{12}\left(1+1+1+\frac{1}{2}\right)=\frac{7}{24}
\end{align*}

\begin{align*}
\text{bc}(d) & =\frac{1}{12}\left(0\right)=0
\end{align*}

\begin{align*}
\text{bc}(e) & =\frac{1}{12}\left(0\right)=0
\end{align*}


\subsubsection{Closeness centrality $C(i)$}
\begin{itemize}
\item $C\left(A\right)=\frac{5-1}{1+1++2+2}=\frac{2}{3}$
\item $C\left(B\right)=\frac{4}{1+1+1+2}=\frac{4}{5}$
\item $C\left(C\right)=\frac{4}{4}=1$
\item $C\left(D\right)=\frac{4}{1+2+2+2}=\frac{4}{7}$
\item $C\left(E\right)=\frac{4}{1+1+2+2}=\frac{2}{3}$
\end{itemize}

\subsubsection{K-shell $k_{s}(i)$}
\begin{itemize}
\item 1-shell = {[}A, B, C, D, E{]}
\item 2-shell = {[}A, B, C, E{]}
\end{itemize}

\subsection{Centrality measures}

\begin{figure}[H]
\begin{centering}
\includegraphics[width=0.6\paperwidth]{ex3/graphs/centrality/_small_cayley_tree}
\par\end{centering}
\caption{Cayley tree}
\end{figure}

\begin{figure}[H]
\begin{centering}
\includegraphics[width=0.6\paperwidth]{ex3/graphs/centrality/_lattice}
\par\end{centering}
\caption{lattice}
\end{figure}

\begin{figure}[H]
\begin{centering}
\includegraphics[width=0.6\paperwidth]{ex3/graphs/centrality/_small_ring}
\par\end{centering}
\caption{Ring}
\end{figure}

\begin{figure}[H]
\begin{centering}
\includegraphics[width=0.6\paperwidth]{ex3/graphs/centrality/_karate_club_network}
\par\end{centering}
\caption{Karate club network}
\end{figure}


\subsection{Visualizations}

\begin{figure}[H]
\begin{centering}
\includegraphics[width=0.6\paperwidth]{ex3/graphs/centrality/_networks_small_cayley_tree}
\par\end{centering}
\caption{Cayley tree}
\end{figure}

\begin{figure}[H]
\begin{centering}
\includegraphics[width=0.6\paperwidth]{ex3/graphs/centrality/_networks_lattice}
\par\end{centering}
\caption{lattice}
\end{figure}

\begin{figure}[H]
\begin{centering}
\includegraphics[width=0.6\paperwidth]{ex3/graphs/centrality/_networks_small_ring}
\par\end{centering}
\caption{Ring}
\end{figure}

\begin{figure}[H]
\begin{centering}
\includegraphics[width=0.6\paperwidth]{ex3/graphs/centrality/_networks_karate_club_network}
\par\end{centering}
\caption{Karate club network}
\end{figure}


\section{PageRank (directed network)}

\subsection{Display the network}

\begin{figure}[H]
\begin{centering}
\includegraphics[width=0.6\paperwidth]{\string"ex3/graphs/page_rank/graphs_uncolored (copy)\string".pdf}
\par\end{centering}
\caption{page rank - test network}
\end{figure}


\subsection{Naive PageRank}

\begin{figure}[H]
\begin{centering}
\includegraphics[width=0.6\paperwidth]{ex3/graphs/page_rank/colored_graph}
\par\end{centering}
\caption{naive page rank - $d=0.85$ $N_{\text{steps}}=10\,000$ on the test
network}
\end{figure}
\begin{figure}[H]
\begin{centering}
\includegraphics[width=0.6\paperwidth]{ex3/graphs/page_rank/check_sanity_random}
\par\end{centering}
\caption{naive page rank - $d=0.85$ $N_{\text{steps}}=10\,000$ on the test
network}
\end{figure}


\subsection{Power iterations PageRank}

\begin{figure}[H]
\begin{centering}
\includegraphics[width=0.6\paperwidth]{ex3/graphs/page_rank/power_iterations}
\par\end{centering}
\caption{power iteration page rank - $d=0.85$ $N_{\text{steps}}=10\,000$
on the test network}
\end{figure}
\begin{figure}[H]
\begin{centering}
\includegraphics[width=0.6\paperwidth]{ex3/graphs/page_rank/check_sanity}
\par\end{centering}
\caption{power iteration page rank - $d=0.85$ $N_{\text{steps}}=10\,000$
on the test network}
\end{figure}


\subsection{Estimating PageRank computation time}
\begin{verbatim}
power iteration (k5 size = 10 ** 3) = 0.7829688139972859 
random walker (k5 size = 10**3) = 92.53090116499516

\end{verbatim}
\selectlanguage{english}%
Then, it we assume that the computation time is linear with respect
to the size of the network, then power iteration should take 20~357.18
secondes ($\sim5$hours) on the full network. The random walk would
take $\sim668.27$ hours.

\subsection{How the network\textquoteright s structure relates to PageRank}

\subsubsection{What is the connection between degree k or in-degree k in and PageRank?}

The degree seems more or less correlated with the page rank. A high
in-degree node has a good probability to yield a good page rank.

\subsubsection{How does PageRank change if the node belongs to a strongly connected
component?}

The page rank improves if the node belongs to a strongly connected
component because it is more likely that the walker will stay in the
component.

\subsubsection{How could this information be used in improving the power iteration
algorithm given in part c)?}

We could use this value to provide a better values for the starting
point of the algorithm. We should take the degree into account, thus
it could converge faster.

\subsection{Dampling factor}

\begin{figure}[H]
\begin{centering}
\includegraphics[width=0.6\paperwidth]{ex3/graphs/page_rank/d_effect}
\par\end{centering}
\caption{Comparaison of different dampling factor}
\end{figure}


\subsubsection{How does the change of d affect the rank of the nodes and the absolute
PageRank values?}

The dampling factor $d$ seems to influence the contrast between high
rankgpage node value and low rankpage node value. If it too small,
then we cannot see very weel how the node of each node is with respect
to the others. If it is too high, then the high page rank value is
kind of ``too saturated'', so you can not make a difference any
more..

\subsection{Page rank wikipedia}

\paragraph{Results}
\begin{itemize}
\item Highest PageRank:
\begin{itemize}
\item 0.03519319071432259 : Graph\_theory
\item 0.020361350619844686 : Social\_network
\item 0.01677151139830182 : Mathematics
\item 0.01646208363207607 : Social\_network\_analysis
\item 0.014703296264824407 : Social\_networking\_service
\end{itemize}
\item Highest In-degree:
\begin{itemize}
\item 82 : Social\_network
\item 73 : Social\_network\_analysis
\item 63 : Small\_world\_experiment
\item 62 : Social\_networking\_service
\item 62 : Orkut
\end{itemize}
\item Highest Out-degree:
\begin{itemize}
\item 140 : Network\_science
\item 82 : Social\_network
\item 73 : Social\_network\_analysis
\item 67 : Small-world\_network
\item 65 : Sexual\_network
\end{itemize}
\end{itemize}

\paragraph{Comment the differences and similarities between the three lists
of most central pages}
\begin{itemize}
\item ``Graph therory'' and ``Mathematics'' have a high page rank, but
they suprisingly don't have belong to the tops highest in-degree.
For example, ``social network'' is in the top three each time.
\item ``Network science'' lead to many pages, but not there are not a
lot of pages that link back to it.
\item ``social network'' and ``social network analysis'' have both correlated
subject, that's why they have a good ranking as it can improve this
way.
\end{itemize}

\end{document}
